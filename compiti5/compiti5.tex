\documentclass{article}

\usepackage{fancyhdr}
\usepackage{extramarks}
\usepackage{amsmath}
\usepackage{amsthm}
\usepackage{amsfonts}
\usepackage{tikz}
\usepackage[plain]{algorithm}
\usepackage{algpseudocode}

% \usepackage{bbold}
\usepackage{dsfont}
\usepackage{braket}
\usepackage{cancel}
\usepackage{booktabs}
\usepackage{graphicx}

\usetikzlibrary{automata,positioning}

%
% Basic Document Settings
%

\topmargin=-0.45in
\evensidemargin=0in
\oddsidemargin=0in
\textwidth=6.5in
\textheight=9.0in
\headsep=0.25in

\linespread{1.1}

\pagestyle{fancy}
\lhead{\hmwkAuthorName}
\chead{\hmwkClass: \hmwkTitle}
\rhead{\firstxmark}
\lfoot{\lastxmark}
\cfoot{\thepage}

\renewcommand\headrulewidth{0.4pt}
\renewcommand\footrulewidth{0.4pt}

\setlength\parindent{0pt}

%
% Create Problem Sections
%

\newcommand{\enterProblemHeader}[1]{
    \nobreak\extramarks{Problema \arabic{#1}}{}
}

\newcommand{\exitProblemHeader}[1]{
    \nobreak\extramarks{Problema \arabic{#1}}{Problema \arabic{#1}}
    \stepcounter{#1}
    \nobreak\extramarks{Problema \arabic{#1}}{}
}

\setcounter{secnumdepth}{0}
\newcounter{partCounter}
\newcounter{homeworkProblemCounter}
\setcounter{homeworkProblemCounter}{1}
\nobreak\extramarks{Problema \arabic{homeworkProblemCounter}}{}\nobreak{}

%
% Homework Problem Environment
%
% This environment takes an optional argument. When given, it will adjust the
% problem counter. This is useful for when the problems given for your
% assignment aren't sequential. See the last 3 problems of this template for an
% example.
%
\newenvironment{homeworkProblem}[1][-1]{
    \ifnum#1>0
        \setcounter{homeworkProblemCounter}{#1}
    \fi
    \section{Problema \arabic{homeworkProblemCounter}}
    \setcounter{partCounter}{1}
    \enterProblemHeader{homeworkProblemCounter}
}{
    \exitProblemHeader{homeworkProblemCounter}
}

%
% Homework Details
%   - Title
%   - Due date
%   - Class
%   - Section/Time
%   - Instructor
%   - Author
%

\newcommand{\hmwkTitle}{Compiti \#4}
\newcommand{\hmwkDueDate}{28 Nov, 2022}
\newcommand{\hmwkClass}{Fisica Teorica della Materia}
% \newcommand{\hmwkClassTime}{Section A}
% \newcommand{\hmwkClassInstructor}{Professor Isaac Newton}
\newcommand{\hmwkAuthorName}{\textbf{Gianluca Zappavigna}}

%
% Title Page
%

% \title{
%     \vspace{2in}
%     \textmd{\textbf{\hmwkClass:\ \hmwkTitle}}\\
%     \normalsize\vspace{0.1in}\small{Due\ on\ \hmwkDueDate\ at 3:10pm}\\
%     \vspace{0.1in}\large{\textit{\hmwkClassInstructor\ \hmwkClassTime}}
%     \vspace{3in}
% }

% \author{\hmwkAuthorName}
% \date{}

\renewcommand{\part}[1]{\textbf{Parte \arabic{partCounter}}\stepcounter{partCounter}\\}

%
% Various Helper Commands
%

% Useful for algorithms
\newcommand{\alg}[1]{\textsc{\bfseries \footnotesize #1}}

% For derivatives
\newcommand{\deriv}[1]{\frac{\mathrm{d}}{\mathrm{d}x} (#1)}

% For partial derivatives
\newcommand{\pderiv}[2]{\frac{\partial}{\partial #1} #2}

% Integral dx
\newcommand{\dx}{\mathrm{d}x}

% Alias for the Solution section header
\newcommand{\solution}{\textbf{\large Solution}}

% Probability commands: Expectation, Variance, Covariance, Bias
\newcommand{\E}{\mathrm{E}}
\newcommand{\Var}{\mathrm{Var}}
\newcommand{\Cov}{\mathrm{Cov}}
\newcommand{\Bias}{\mathrm{Bias}}

\newcommand{\pvec}[1]{\vec{#1}\mkern2mu\vphantom{#1}}

\begin{document}
% \maketitle
% \pagebreak

\begin{homeworkProblem}
    \part

    Le funzioni d'onda $\psi_S(\vec{r}_1, \vec{r}_2)$ e $\psi_T(\vec{r}_1, \vec{r}_2)$ sono reali.

    \begin{align*}
        \psi_S(\vec{r}_1, \vec{r}_2) = C_S (\psi_a(\vec{r}_1)\psi_b(\vec{r}_2) + \psi_b(\vec{r}_1)\psi_a(\vec{r}_2)) \\
        \psi_T(\vec{r}_1, \vec{r}_2) = C_T (\psi_a(\vec{r}_1)\psi_b(\vec{r}_2) - \psi_b(\vec{r}_1)\psi_a(\vec{r}_2))
    \end{align*}

    \begin{align*}
        \Braket{\psi_S | \psi_S} = 1 && \Braket{\psi_T | \psi_T} = 1
    \end{align*}

    \begin{align*}
        \int d^3 r_1 d^3 r_2 \psi_S^*(\vec{r}_1, \vec{r}_2) \psi_S(\vec{r}_1, \vec{r}_2) =
        \\
        \int d^3 r_1 d^3 r_2 \psi_S(\vec{r}_1, \vec{r}_2) \psi_S(\vec{r}_1, \vec{r}_2) =
        \\
        C_S^2 \int d^3 r_1 d^3 r_2 \left(\psi_a(\vec{r}_1)\psi_b(\vec{r}_2) + \psi_b(\vec{r}_1)\psi_a(\vec{r}_2)\right)
            \left(\psi_a(\vec{r}_1)\psi_b(\vec{r}_2) + \psi_b(\vec{r}_1)\psi_a(\vec{r}_2)\right) =
        \\
        = C_S^2 \left(\int d^3 r_1 d^3 r_2 \psi_a^2(\vec{r}_1)\psi_b^2(\vec{r}_2)
        + \int d^3 r_1 d^3 r_2 \psi_b^2(\vec{r}_1)\psi_a^2(\vec{r}_2)
        + 2\int d^3 r_1 d^3 r_2 \psi_a(\vec{r}_1)\psi_b(\vec{r}_1)\psi_a(\vec{r}_2)\psi_b(\vec{r}_2) \right)
    \end{align*}

    Le funzioni (...) sono già normalizzate, quindi i primi due integrali danno 1.

    \begin{align*}
        C_S^2 \left(2 + 2\int d^3 r_1  \psi_a(\vec{r}_1)\psi_b(\vec{r}_1) \int d^3 r_2 \psi_a(\vec{r}_2)\psi_b(\vec{r}_2) \right) =
        C_S^2 \left(2 + 2S^2 \right)
    \end{align*}

    Dove abbiamo definito
    \begin{equation*}
        S \equiv \int d^3 r \psi_a(\vec{r})\psi_b(\vec{r})
    \end{equation*}

    Per $\psi_T$ i calcoli sono quasi identici, l'unica differenza è che il terzo integrale ha il meno, quindi si trova
    \begin{equation*}
        \int d^3 r_1 d^3 r_2 \psi_T(\vec{r}_1, \vec{r}_2) \psi_T(\vec{r}_1, \vec{r}_2) =
        C_T^2(2 - 2S^2)
    \end{equation*}


    \begin{align*}
        C_S = \frac{1}{\sqrt{2(1 + S^2)}} && C_T = \frac{1}{\sqrt{2(1 - S^2)}}
    \end{align*}

    \part

    \begin{align*}
        S = \int d^3 r \psi_a(\vec{r})\psi_b(\vec{r}) =
        \int d^3 r \left(\frac{1}{\sqrt{\pi a^3}} \exp \left(- \frac{\left|\vec{r} - \vec{r}_a\right| }{a}\right)\right) \left(\frac{1}{\sqrt{\pi a^3}} \exp \left(- \frac{\left|\vec{r} - \vec{r}_b\right| }{a}\right)\right) =
        \\
        = \frac{1}{\pi a^3} \int d^3 r  \exp \left(- \frac{\left|\vec{r} - \vec{r}_a\right| + \left|\vec{r} - \vec{r}_b\right|}{a}\right)
    \end{align*}

    \begin{equation*}
        \begin{cases}
            \xi = \frac{\left|\vec{r} - \vec{r}_a\right| + \left|\vec{r} - \vec{r}_b\right|}{R} \\
            \eta = \frac{\left|\vec{r} - \vec{r}_a\right| - \left|\vec{r} - \vec{r}_b\right|}{R} \\
            \phi = \mathrm{prova}
        \end{cases}
    \end{equation*}
    dove $R \equiv \left|\vec{r}_a - \vec{r}_b\right|$

    Lo Jacobiano di questa trasformazione è $\frac{R^3}{8}(\xi^2 - \eta^2)$

    \begin{align*}
        \frac{1}{\pi a^3} \int d^3 r  \exp \left(- \frac{\left|\vec{r} - \vec{r}_a\right| + \left|\vec{r} - \vec{r}_b\right|}{a}\right)
        = \frac{1}{\pi a^3} \int d^3 r  \exp \left(- \frac{R}{a} \frac{\left|\vec{r} - \vec{r}_a\right| + \left|\vec{r} - \vec{r}_b\right|}{R}\right) =
        \\
        = \frac{1}{\pi a^3} \int_0^{2\pi} d\phi \int_{-1}^1 d\eta \int_1^\infty d\xi \exp \left(- \frac{R}{a} \xi \right) \frac{R^3}{8}(\xi^2 - \eta^2) =
        \\
        = \frac{R^3}{8 \pi a^3} \left(\int_0^{2\pi} d\phi \int_{-1}^1 d\eta \int_1^\infty d\xi \exp \left(- \frac{R}{a} \xi \right) \xi^2
        - \int_0^{2\pi} d\phi \int_{-1}^1 d\eta\, \eta^2 \int_1^\infty d\xi \exp \left(- \frac{R}{a} \xi \right) \right) = \\
        \\
        = \frac{R^3}{8 \pi a^3} \left(2 \pi \cdot 2 \cdot \int_1^\infty d\xi \exp \left(- \frac{R}{a} \xi \right) \xi^2
        - 2 \pi \cdot \left[\frac{\eta^3}{3}\right]_{-1}^1 \left[\frac{\exp \left(- \frac{R}{a} \xi \right)}{- \frac{R}{a}} \right]_1^\infty\right) =
        \\
        = \frac{R^3}{8 \pi a^3} \left(4 \pi \int_1^\infty d\xi \exp \left(- \frac{R}{a} \xi \right) \xi^2
        - 2 \pi \cdot \frac{2}{3} \cdot \frac{\exp \left(- \frac{R}{a} \right)}{\frac{R}{a}}\right) =
        \\
        = \frac{R^3}{8 \pi a^3} \left(4 \pi \cdot \frac{2! \exp\left(-\frac{R}{a}\right)}{\left(\frac{R}{a}\right)^3} \sum_{k=0}^2 \frac{\left(\frac{R}{a}\right)^k}{k!}
        - \frac{4 \pi}{3} \cdot \frac{\exp \left(- \frac{R}{a} \right)}{\frac{R}{a}}\right) =
        \\
        = \frac{R^3 \exp\left(-\frac{R}{a}\right)}{2 a^3} \left( \frac{2}{\left(\frac{R}{a}\right)^3} \sum_{k=0}^2 \frac{\left(\frac{R}{a}\right)^k}{k!}
        - \frac{1}{3} \frac{1}{\frac{R}{a}}\right) =
        \\
        = \exp\left(-\frac{R}{a}\right) \left( \sum_{k=0}^2 \frac{\left(\frac{R}{a}\right)^k}{k!}
        - \frac{1}{6}\left(\frac{R}{a}\right)^2 \right) =
        \\
        = \exp\left(-\frac{R}{a}\right) \left( 1 + \frac{1}{2} \left(\frac{R}{a}\right) + \frac{1}{6} \left(\frac{R}{a}\right)^2
        - \frac{1}{6}\left(\frac{R}{a}\right)^2 \right) =
        \\
        = \exp\left(-\frac{R}{a}\right) \left( 1 + \frac{1}{2} \left(\frac{R}{a}\right) \right)
    \end{align*}

    \part

    \begin{equation*}
        \frac{\partial}{\partial r_i} \left|\vec{r} - \vec{r}_a\right| = \frac{r_i - r_{a_i}}{\left|\vec{r} - \vec{r}_a\right|}
        = \frac{\left(\vec{r} - \vec{r}_a\right)_i}{\left|\vec{r} - \vec{r}_a\right|}
    \end{equation*}

    \begin{equation*}
        \nabla\left|\vec{r} - \vec{r}_a\right| = \frac{\vec{r} - \vec{r}_a}{\left|\vec{r} - \vec{r}_a\right|}
    \end{equation*}


    \begin{align*}
        \nabla \exp \left(- \frac{\left|\vec{r} - \vec{r}_a\right|}{a}\right) = \exp \left(- \frac{\left|\vec{r} - \vec{r}_a\right|}{a}\right) \nabla \left(- \frac{\left|\vec{r} - \vec{r}_a\right|}{a}\right) = \\
        = -\frac{1}{a} \exp \left(- \frac{\left|\vec{r} - \vec{r}_a\right|}{a}\right) \frac{\vec{r} - \vec{r}_a}{\left|\vec{r} - \vec{r}_a\right|}
    \end{align*}


    \begin{align*}
        \frac{\partial}{\partial r_i} \left(-\frac{1}{a} \exp \left(- \frac{\left|\vec{r} - \vec{r}_a\right|}{a}\right) \frac{(\vec{r} - \vec{r}_a)_i}{\left|\vec{r} - \vec{r}_a\right|}\right) =
        \\
        = -\frac{1}{a} \left[\frac{(\vec{r} - \vec{r}_a)_i}{\left|\vec{r} - \vec{r}_a\right|} \frac{\partial}{\partial r_i} \exp \left(- \frac{\left|\vec{r} - \vec{r}_a\right|}{a}\right) +
        \exp \left(- \frac{\left|\vec{r} - \vec{r}_a\right|}{a}\right) \frac{\partial}{\partial r_i} \frac{(\vec{r} - \vec{r}_a)_i}{\left|\vec{r} - \vec{r}_a\right|}
        \right]
        \\
        = -\frac{1}{a} \exp \left(- \frac{\left|\vec{r} - \vec{r}_a\right|}{a}\right) \left[\frac{(\vec{r} - \vec{r}_a)_i}{\left|\vec{r} - \vec{r}_a\right|} \frac{\partial}{\partial r_i} \left(- \frac{\left|\vec{r} - \vec{r}_a\right|}{a}\right)
        + \frac{\partial}{\partial r_i} \frac{(\vec{r} - \vec{r}_a)_i}{\left|\vec{r} - \vec{r}_a\right|}
        \right]
        \\
        = -\frac{1}{a} \exp \left(- \frac{\left|\vec{r} - \vec{r}_a\right|}{a}\right) \left[ -\frac{1}{a} \frac{(\vec{r} - \vec{r}_a)_i}{\left|\vec{r} - \vec{r}_a\right|} \frac{(\vec{r} - \vec{r}_a)_i}{\left|\vec{r} - \vec{r}_a\right|}
        + \frac{1 \cdot \left|\vec{r} - \vec{r}_a\right| - (\vec{r} - \vec{r}_a)_i \cdot \frac{(\vec{r} - \vec{r}_a)_i}{\left|\vec{r} - \vec{r}_a\right|} }{\left|\vec{r} - \vec{r}_a\right|^2}
        \right]
        \\
        = -\frac{1}{a} \exp \left(- \frac{\left|\vec{r} - \vec{r}_a\right|}{a}\right) \left[ -\frac{1}{a} \frac{(\vec{r} - \vec{r}_a)^2_i}{\left|\vec{r} - \vec{r}_a\right|^2}
        + \frac{1}{\left|\vec{r} - \vec{r}_a\right|} -\frac{(\vec{r} - \vec{r}_a)^2_i}{\left|\vec{r} - \vec{r}_a\right|^3}
        \right]
    \end{align*}


    \begin{align*}
        -\frac{1}{a} \exp \left(- \frac{\left|\vec{r} - \vec{r}_a\right|}{a}\right) \sum_i \left[ -\frac{1}{a} \frac{(\vec{r} - \vec{r}_a)^2_i}{\left|\vec{r} - \vec{r}_a\right|^2}
        + \frac{1}{\left|\vec{r} - \vec{r}_a\right|} -\frac{(\vec{r} - \vec{r}_a)^2_i}{\left|\vec{r} - \vec{r}_a\right|^3}
        \right] =
        \\
        = -\frac{1}{a} \exp \left(- \frac{\left|\vec{r} - \vec{r}_a\right|}{a}\right) \left[ -\frac{1}{a} \frac{\left|\vec{r} - \vec{r}_a\right|^2}{\left|\vec{r} - \vec{r}_a\right|^2}
        + \frac{3}{\left|\vec{r} - \vec{r}_a\right|} -\frac{\left|\vec{r} - \vec{r}_a\right|^2}{\left|\vec{r} - \vec{r}_a\right|^3}
        \right] =
        \\
        = -\frac{1}{a} \exp \left(- \frac{\left|\vec{r} - \vec{r}_a\right|}{a}\right) \left( -\frac{1}{a} + \frac{2}{\left|\vec{r} - \vec{r}_a\right|} \right) =
        \\
        = \exp \left(- \frac{\left|\vec{r} - \vec{r}_a\right|}{a}\right) \left(\frac{1}{a^2} - \frac{2}{a \left|\vec{r} - \vec{r}_a\right|} \right)
    \end{align*}

    \begin{equation*}
        V(\vec{r}_1, \vec{r}_2) \equiv e^2 \left(
            \frac{1}{\left|\vec{r}_1 - \vec{r}_2\right|}
            - \frac{1}{\left|\vec{r}_1 - \vec{r}_a\right|}
            - \frac{1}{\left|\vec{r}_1 - \vec{r}_b\right|}
            - \frac{1}{\left|\vec{r}_2 - \vec{r}_a\right|}
            - \frac{1}{\left|\vec{r}_2 - \vec{r}_b\right|}
            + \frac{1}{\left|\vec{r}_a - \vec{r}_b\right|}
        \right)
    \end{equation*}

    \begin{align*}
        \int d^3 r_1 d^3 r_2 \psi_S(\vec{r}_1, \vec{r}_2) \left(-\frac{\hbar^2}{2m}(\nabla^2_1 + \nabla^2_2) + V(\vec{r}_1, \vec{r}_2)\right) \psi_S(\vec{r}_1, \vec{r}_2) =
        \\
        \int d^3 r_1 d^3 r_2 \psi_S(\vec{r}_1, \vec{r}_2) \left(-\frac{\hbar^2}{2m}(\nabla^2_1 + \nabla^2_2)\right) \psi_S(\vec{r}_1, \vec{r}_2)
        + \int d^3 r_1 d^3 r_2 V(\vec{r}_1, \vec{r}_2) \psi_S(\vec{r}_1, \vec{r}_2) \psi_S(\vec{r}_1, \vec{r}_2)
    \end{align*}

    \begin{align*}
        C_S^2 \int d^3 r_1 d^3 r_2 \left(\psi_a(\vec{r}_1)\psi_b(\vec{r}_2) + \psi_b(\vec{r}_1)\psi_a(\vec{r}_2)\right) \left(-\frac{\hbar^2}{2m}(\nabla^2_1 + \nabla^2_2) \right) \left(\psi_a(\vec{r}_1)\psi_b(\vec{r}_2) + \psi_b(\vec{r}_1)\psi_a(\vec{r}_2)\right) =
        \\
        = \int d^3 r_1 d^3 r_2 \left(\psi_a(\vec{r}_1)\psi_b(\vec{r}_2) + \psi_b(\vec{r}_1)\psi_a(\vec{r}_2)\right) \times
        \\
        \times \left(-\frac{\hbar^2}{2m}(\psi_b(\vec{r}_2) \nabla^2_1 \psi_a(\vec{r}_1) + \psi_a(\vec{r}_2) \nabla^2_1 \psi_b(\vec{r}_1)
         +\psi_b(\vec{r}_1) \nabla^2_2 \psi_a(\vec{r}_2) + \psi_a(\vec{r}_1) \nabla^2_2 \psi_b(\vec{r}_2)) \right)
        % \\
        % + \int d^3 r_1 d^3 r_2 \left(\psi_a(\vec{r}_1)\psi_b(\vec{r}_2) + \psi_b(\vec{r}_1)\psi_a(\vec{r}_2)\right) V \left(\psi_a(\vec{r}_1)\psi_b(\vec{r}_2) + \psi_b(\vec{r}_1)\psi_a(\vec{r}_2)\right)
    \end{align*}

    Consideriamo solo la parentesi con il laplaciani

    \begin{align*}
        -\frac{\hbar^2}{2m}(\psi_b(\vec{r}_2) \nabla^2_1 \psi_a(\vec{r}_1) + \psi_a(\vec{r}_2) \nabla^2_1 \psi_b(\vec{r}_1)
         +\psi_b(\vec{r}_1) \nabla^2_2 \psi_a(\vec{r}_2) + \psi_a(\vec{r}_1) \nabla^2_2 \psi_b(\vec{r}_2)) =
         \\
         = -\frac{\hbar^2}{2m} \left( \psi_b(\vec{r}_2) \psi_a(\vec{r}_1) \left(\frac{1}{a^2} - \frac{2}{a\left|\vec{r}_1  - \vec{r}_a\right|}\right)
         + \psi_a(\vec{r}_2) \psi_b(\vec{r}_1) \left(\frac{1}{a^2} - \frac{2}{a\left|\vec{r}_1  - \vec{r}_b\right|}\right) \right.
         \\
         +\left. \psi_b(\vec{r}_1) \psi_a(\vec{r}_2) \left(\frac{1}{a^2} - \frac{2}{a\left|\vec{r}_2 - \vec{r}_a\right|}\right)
         + \psi_a(\vec{r}_1) \psi_b(\vec{r}_2) \left(\frac{1}{a^2} - \frac{2}{a\left|\vec{r}_2 - \vec{r}_b\right|}\right)\right) =
         \\
         = -\frac{\hbar^2}{2m} \left( \psi_a(\vec{r}_1) \psi_b(\vec{r}_2) \left(\frac{2}{a^2} - \frac{2}{a\left|\vec{r}_1  - \vec{r}_a\right|} - \frac{2}{a\left|\vec{r}_2  - \vec{r}_b\right|}\right)
         + \psi_b(\vec{r}_1) \psi_a(\vec{r}_2) \left(\frac{2}{a^2} - \frac{2}{a\left|\vec{r}_1  - \vec{r}_b\right|} - \frac{2}{a\left|\vec{r}_2 - \vec{r}_a\right|}\right)\right) =
    \end{align*}

    Moltiplicando per $\left(\psi_a(\vec{r}_1)\psi_b(\vec{r}_2) + \psi_b(\vec{r}_1)\psi_a(\vec{r}_2)\right)$ si trova

    \begin{align*}
        = -\frac{\hbar^2}{2m} \left( \psi_a(\vec{r}_1)^2 \psi_b(\vec{r}_2)^2 \left(\frac{2}{a^2} - \frac{2}{a\left|\vec{r}_1  - \vec{r}_a\right|} - \frac{2}{a\left|\vec{r}_2  - \vec{r}_b\right|}\right)
        + \psi_b(\vec{r}_1)^2 \psi_a(\vec{r}_2)^2 \left(\frac{2}{a^2} - \frac{2}{a\left|\vec{r}_1  - \vec{r}_b\right|} - \frac{2}{a\left|\vec{r}_2 - \vec{r}_a\right|}\right) \right.
        \\
        + \left. \psi_a(\vec{r}_1) \psi_b(\vec{r}_1) \psi_a(\vec{r}_2) \psi_b(\vec{r}_2) \left(
        \frac{4}{a^2} - \frac{2}{a\left|\vec{r}_1  - \vec{r}_a\right|} - \frac{2}{a\left|\vec{r}_2  - \vec{r}_b\right|}  - \frac{2}{a\left|\vec{r}_1  - \vec{r}_b\right|} - \frac{2}{a\left|\vec{r}_2 - \vec{r}_a\right|}
        \right)
        \right)
        \\
    \end{align*}

    \begin{align*}
        \int d^3 r_1 d^3 r_2 V(\vec{r}_1, \vec{r}_2) \psi_S(\vec{r}_1, \vec{r}_2) \psi_S(\vec{r}_1, \vec{r}_2) = \\
        C_S^2 \int d^3 r_1 d^3 r_2 V(\vec{r}_1, \vec{r}_2) \left(\psi_a(\vec{r}_1)\psi_b(\vec{r}_2) + \psi_b(\vec{r}_1)\psi_a(\vec{r}_2)\right) \cdot \left(\psi_a(\vec{r}_1)\psi_b(\vec{r}_2) + \psi_b(\vec{r}_1)\psi_a(\vec{r}_2)\right) = \\
        C_S^2 \int d^3 r_1 d^3 r_2 V(\vec{r}_1, \vec{r}_2) \left(\psi^2_a(\vec{r}_1) \psi^2_b(\vec{r}_2) + \psi^2_b(\vec{r}_1) \psi^2_a(\vec{r}_2) + 2 \psi_a(\vec{r}_1)\psi_b(\vec{r}_2) \psi_b(\vec{r}_1)\psi_a(\vec{r}_2)\right) = \\
    \end{align*}

    Dato che 

    \begin{align*}
        &\int d^3 r_1 d^3 r_2  \psi_a(\vec{r}_1)^2 \psi_b(\vec{r}_2)^2 \left(\frac{2}{a^2} - \frac{2}{a\left|\vec{r}_1  - \vec{r}_a\right|} - \frac{2}{a\left|\vec{r}_2  - \vec{r}_b\right|}\right)
        = \int d^3 r_1 d^3 r_2  \psi_a(\vec{r}_2)^2 \psi_b(\vec{r}_1)^2  \left(\frac{2}{a^2} - \frac{2}{a\left|\vec{r}_2  - \vec{r}_a\right|} - \frac{2}{a\left|\vec{r}_1 - \vec{r}_b\right|}\right)
        \\
        \vspace{1em}
        \\
        &\int d^3 r_1 d^3 r_2  \psi_a(\vec{r}_1)\psi_b(\vec{r}_2) \psi_b(\vec{r}_1)\psi_a(\vec{r}_2) \left(\frac{2}{a^2} - \frac{2}{a\left|\vec{r}_1  - \vec{r}_a\right|} - \frac{2}{a\left|\vec{r}_2  - \vec{r}_b\right|}\right)  = \\
        &= \int d^3 r_1 d^3 r_2  \psi_a(\vec{r}_1)\psi_b(\vec{r}_2) \psi_b(\vec{r}_1)\psi_a(\vec{r}_2)  \left(\frac{2}{a^2} - \frac{2}{a\left|\vec{r}_2  - \vec{r}_a\right|} - \frac{2}{a\left|\vec{r}_1 - \vec{r}_b\right|}\right)
        \\
        \vspace{1em}
        \\
        & \int d^3 r_1 d^3 r_2 V(\vec{r}_1, \vec{r}_2) \left(\psi^2_a(\vec{r}_1) \psi^2_b(\vec{r}_2)\right)  = \int d^3 r_1 d^3 r_2 V(\vec{r}_1, \vec{r}_2) \left( \psi^2_b(\vec{r}_1) \psi^2_a(\vec{r}_2) \right)
    \end{align*}


\end{homeworkProblem}
\end{document}
