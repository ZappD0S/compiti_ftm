\documentclass{article}

\usepackage{fancyhdr}
\usepackage{extramarks}
\usepackage{amsmath}
\usepackage{amsthm}
\usepackage{amsfonts}
\usepackage{tikz}
\usepackage[plain]{algorithm}
\usepackage{algpseudocode}

% \usepackage{bbold}
\usepackage{dsfont}
\usepackage{braket}
\usepackage{cancel}
\usepackage{booktabs}
\usepackage{graphicx}

\usetikzlibrary{automata,positioning}

%
% Basic Document Settings
%

\topmargin=-0.45in
\evensidemargin=0in
\oddsidemargin=0in
\textwidth=6.5in
\textheight=9.0in
\headsep=0.25in

\linespread{1.1}

\pagestyle{fancy}
\lhead{\hmwkAuthorName}
\chead{\hmwkClass: \hmwkTitle}
\rhead{\firstxmark}
\lfoot{\lastxmark}
\cfoot{\thepage}

\renewcommand\headrulewidth{0.4pt}
\renewcommand\footrulewidth{0.4pt}

\setlength\parindent{0pt}

%
% Create Problem Sections
%

\newcommand{\enterProblemHeader}[1]{
    \nobreak\extramarks{Problema \arabic{#1}}{}
}

\newcommand{\exitProblemHeader}[1]{
    \nobreak\extramarks{Problema \arabic{#1}}{Problema \arabic{#1}}
    \stepcounter{#1}
    \nobreak\extramarks{Problema \arabic{#1}}{}
}

\setcounter{secnumdepth}{0}
\newcounter{partCounter}
\newcounter{homeworkProblemCounter}
\setcounter{homeworkProblemCounter}{1}
\nobreak\extramarks{Problema \arabic{homeworkProblemCounter}}{}\nobreak{}

%
% Homework Problem Environment
%
% This environment takes an optional argument. When given, it will adjust the
% problem counter. This is useful for when the problems given for your
% assignment aren't sequential. See the last 3 problems of this template for an
% example.
%
\newenvironment{homeworkProblem}[1][-1]{
    \ifnum#1>0
        \setcounter{homeworkProblemCounter}{#1}
    \fi
    \section{Problema \arabic{homeworkProblemCounter}}
    \setcounter{partCounter}{1}
    \enterProblemHeader{homeworkProblemCounter}
}{
    \exitProblemHeader{homeworkProblemCounter}
}

%
% Homework Details
%   - Title
%   - Due date
%   - Class
%   - Section/Time
%   - Instructor
%   - Author
%

\newcommand{\hmwkTitle}{Compiti \#4}
\newcommand{\hmwkDueDate}{21 Nov, 2022}
\newcommand{\hmwkClass}{Fisica Teorica della Materia}
% \newcommand{\hmwkClassTime}{Section A}
% \newcommand{\hmwkClassInstructor}{Professor Isaac Newton}
\newcommand{\hmwkAuthorName}{\textbf{Gianluca Zappavigna}}

%
% Title Page
%

% \title{
%     \vspace{2in}
%     \textmd{\textbf{\hmwkClass:\ \hmwkTitle}}\\
%     \normalsize\vspace{0.1in}\small{Due\ on\ \hmwkDueDate\ at 3:10pm}\\
%     \vspace{0.1in}\large{\textit{\hmwkClassInstructor\ \hmwkClassTime}}
%     \vspace{3in}
% }

% \author{\hmwkAuthorName}
% \date{}

\renewcommand{\part}[1]{\textbf{Parte \arabic{partCounter}}\stepcounter{partCounter}\\}

%
% Various Helper Commands
%

% Useful for algorithms
\newcommand{\alg}[1]{\textsc{\bfseries \footnotesize #1}}

% For derivatives
\newcommand{\deriv}[1]{\frac{\mathrm{d}}{\mathrm{d}x} (#1)}

% For partial derivatives
\newcommand{\pderiv}[2]{\frac{\partial}{\partial #1} #2}

% Integral dx
\newcommand{\dx}{\mathrm{d}x}

% Alias for the Solution section header
\newcommand{\solution}{\textbf{\large Solution}}

% Probability commands: Expectation, Variance, Covariance, Bias
\newcommand{\E}{\mathrm{E}}
\newcommand{\Var}{\mathrm{Var}}
\newcommand{\Cov}{\mathrm{Cov}}
\newcommand{\Bias}{\mathrm{Bias}}

\newcommand{\pvec}[1]{\vec{#1}\mkern2mu\vphantom{#1}}

\begin{document}
% \maketitle
% \pagebreak

\begin{homeworkProblem}
    \part

    \begin{equation*}
        \left[\hat{n}(\vec{r}, t), \hat{n}(\pvec{r}', t)\right] = 0
    \end{equation*}

    \begin{align*}
        &\left(\sum_\sigma \hat{\psi}_\sigma^\dagger(\vec{r}, t)\hat{\psi}_\sigma(\vec{r}, t)\right)
        \left(\sum_{\sigma'} \hat{\psi}_{\sigma'}^\dagger(\pvec{r}', t)\hat{\psi}_{\sigma'}(\pvec{r}', t)\right)
        - \left(\sum_{\sigma'} \hat{\psi}_{\sigma'}^\dagger(\pvec{r}', t)\hat{\psi}_{\sigma'}(\pvec{r}', t)\right)
        \left(\sum_\sigma \hat{\psi}_\sigma^\dagger(\vec{r}, t)\hat{\psi}_\sigma(\vec{r}, t)\right) =
        \\
        &= \sum_{\sigma, \sigma'} \hat{\psi}_\sigma^\dagger(\vec{r}, t)\hat{\psi}_\sigma(\vec{r}, t)\hat{\psi}_{\sigma'}^\dagger(\pvec{r}', t)\hat{\psi}_{\sigma'}(\pvec{r}', t)
        - \sum_{\sigma, \sigma'} \hat{\psi}_{\sigma'}^\dagger(\pvec{r}', t)\hat{\psi}_{\sigma'}(\pvec{r}', t) \hat{\psi}_\sigma^\dagger(\vec{r}, t)\hat{\psi}_\sigma(\vec{r}, t) =
        \\
        &= \sum_{\sigma, \sigma'} \left(\hat{\psi}_\sigma^\dagger(\vec{r}, t)\hat{\psi}_\sigma(\vec{r}, t)\hat{\psi}_{\sigma'}^\dagger(\pvec{r}', t)\hat{\psi}_{\sigma'}(\pvec{r}', t)\right.\\
        &\phantom{= \sum_{\sigma, \sigma'}\left(\right.} -\hat{\psi}_\sigma^\dagger(\vec{r}, t) \hat{\psi}_{\sigma'}^\dagger(\pvec{r}', t)\hat{\psi}_{\sigma}(\vec{r}, t)\hat{\psi}_{\sigma'}(\pvec{r}', t) \\
        &\phantom{= \sum_{\sigma, \sigma'}\left(\right.} +\hat{\psi}_\sigma^\dagger(\vec{r}, t)\hat{\psi}_{\sigma'}^\dagger(\pvec{r}', t)\hat{\psi}_{\sigma}(\vec{r}, t)\hat{\psi}_{\sigma'}(\pvec{r}', t) \\
        &\phantom{= \sum_{\sigma, \sigma'}\left(\right.}\left. -\hat{\psi}_{\sigma'}^\dagger(\pvec{r}', t)\hat{\psi}_{\sigma'}(\pvec{r}', t) \hat{\psi}_\sigma^\dagger(\vec{r}, t)\hat{\psi}_\sigma(\vec{r}, t)\right)
        \\
        % &\phantom{= \sum_{\sigma, \sigma'}}\left. -\hat{\psi}_{\sigma'}^\dagger(\pvec{r}', t)\hat{\psi}_{\sigma'}(\pvec{r}', t) \hat{\psi}_\sigma^\dagger(\vec{r}, t)\hat{\psi}_\sigma(\vec{r}, t)\right) = \\
        % &\phantom{= \sum_{\sigma, \sigma'}}\left. -\hat{\psi}_{\sigma'}^\dagger(\pvec{r}', t)\hat{\psi}_{\sigma'}(\pvec{r}', t) \hat{\psi}_\sigma^\dagger(\vec{r}, t)\hat{\psi}_\sigma(\vec{r}, t)\right) = \\
    \end{align*}

    \begin{align*}
        \left[\hat{\psi}_\sigma(\vec{r}, t), \hat{\psi}_{\sigma'}(\pvec{r}', t)\right] = 0
        &&
        \left[\hat{\psi}_\sigma^\dagger(\vec{r}, t), \hat{\psi}_{\sigma'}^\dagger(\pvec{r}', t)\right] = 0
    \end{align*}

    \begin{align*}
        &\sum_{\sigma, \sigma'} \left(\hat{\psi}_\sigma^\dagger(\vec{r}, t)\hat{\psi}_\sigma(\vec{r}, t)\hat{\psi}_{\sigma'}^\dagger(\pvec{r}', t)\hat{\psi}_{\sigma'}(\pvec{r}', t)\right.\\
        &\phantom{= \sum_{\sigma, \sigma'}} -\hat{\psi}_\sigma^\dagger(\vec{r}, t) \hat{\psi}_{\sigma'}^\dagger(\pvec{r}', t)\hat{\psi}_{\sigma}(\vec{r}, t)\hat{\psi}_{\sigma'}(\pvec{r}', t) \\
        &\phantom{= \sum_{\sigma, \sigma'}} +\hat{\psi}_{\sigma'}^\dagger(\pvec{r}', t)\hat{\psi}_\sigma^\dagger(\vec{r}, t)\hat{\psi}_{\sigma'}(\pvec{r}', t)\hat{\psi}_{\sigma}(\vec{r}, t) \\
        &\phantom{= \sum_{\sigma, \sigma'}}\left. -\hat{\psi}_{\sigma'}^\dagger(\pvec{r}', t)\hat{\psi}_{\sigma'}(\pvec{r}', t) \hat{\psi}_\sigma^\dagger(\vec{r}, t)\hat{\psi}_\sigma(\vec{r}, t)\right)
        \\
        &= \sum_{\sigma, \sigma'} \hat{\psi}_\sigma^\dagger(\vec{r}, t)\left[\hat{\psi}_\sigma(\vec{r}, t), \hat{\psi}_{\sigma'}^\dagger(\pvec{r}', t)\right]\hat{\psi}_{\sigma'}(\pvec{r}', t)
        + \hat{\psi}_{\sigma'}^\dagger(\pvec{r}', t)\left[\hat{\psi}_\sigma^\dagger(\vec{r}, t), \hat{\psi}_{\sigma'}(\pvec{r}', t)\right]\hat{\psi}_\sigma(\vec{r}, t)
        \\
        &= \sum_{\sigma, \sigma'} \hat{\psi}_\sigma^\dagger(\vec{r}, t)\left[\hat{\psi}_\sigma(\vec{r}, t), \hat{\psi}_{\sigma'}^\dagger(\pvec{r}', t)\right]\hat{\psi}_{\sigma'}(\pvec{r}', t)
        - \hat{\psi}_{\sigma'}^\dagger(\pvec{r}', t)\left[\hat{\psi}_{\sigma'}(\pvec{r}', t), \hat{\psi}_\sigma^\dagger(\vec{r}, t)\right]\hat{\psi}_\sigma(\vec{r}, t)
        \\
        &= \sum_{\sigma, \sigma'} \delta_{\sigma \sigma'}\hat{\psi}_\sigma^\dagger(\vec{r}, t)\hat{\psi}_{\sigma'}(\pvec{r}', t)
        - \delta_{\sigma \sigma'}\hat{\psi}_{\sigma'}^\dagger(\pvec{r}', t)\hat{\psi}_\sigma(\vec{r}, t) =
        \\
        &= \sum_\sigma \hat{\psi}_\sigma^\dagger(\vec{r}, t)\hat{\psi}_\sigma(\vec{r}, t)
        - \hat{\psi}_\sigma^\dagger(\vec{r}, t)\hat{\psi}_\sigma(\vec{r}, t) = 0
    \end{align*}

    \begin{align*}
        \left[\hat{\psi}_\sigma(\vec{r}, t), \hat{\psi}_{\sigma'}^\dagger(\pvec{r}', t)\right] =
        \left[\hat{\psi}_{\sigma'}(\pvec{r}', t), \hat{\psi}_\sigma^\dagger(\vec{r}, t)\right] = \delta_{\sigma \sigma'}
    \end{align*}

    \part

    \begin{equation*}
        \hat{H} = \hat{T} + \hat{V}
    \end{equation*}

    \begin{equation*}
        \hat{T} = \sum_\sigma \int_\Omega d^3r \hat{\psi}_\sigma^\dagger(\vec{r})\left(-\frac{\hbar^2}{2m} \nabla^2\right) \hat{\psi}_\sigma(\vec{r})
    \end{equation*}

    \begin{equation*}
        \hat{V} = \frac{1}{2} \sum_{\sigma, \sigma'} \int_\Omega d^3r d^3r' v(\vec{r} - \pvec{r}') \hat{\psi}_\sigma^\dagger(\vec{r})\hat{\psi}_{\sigma'}^\dagger(\pvec{r}')\hat{\psi}_{\sigma'}(\pvec{r}')\hat{\psi}_\sigma(\vec{r})
    \end{equation*}

    Calcoliamo $\partial_t \hat{\psi}^\dagger_\sigma(\vec{r}, t)$ utilizzando l'eq. di Heisenberg
    \begin{equation*}
        \partial_t \hat{\psi}^\dagger_\sigma(\vec{r}, t) = \frac{i}{\hbar}\left[\hat{H}, \hat{\psi}_\sigma^\dagger(\vec{r}, t)\right]
    \end{equation*}
    Possiamo calcolare il commutatore in due pezzi, dato che
    \begin{equation*}
        \left[\hat{H}, \hat{\psi}_\sigma^\dagger(\vec{r}, t)\right] =
         \left[\hat{T}, \hat{\psi}_\sigma^\dagger(\vec{r}, t)\right]
        + \left[\hat{V}, \hat{\psi}_\sigma^\dagger(\vec{r}, t)\right]
    \end{equation*}

    Per il primo troviamo
    \begin{align*}
        \hat{T} = \int_\Omega d^3r \hat{\psi}_\sigma^\dagger(\vec{r})\left(-\frac{\hbar^2}{2m} \nabla^2\right) \hat{\psi}_\sigma(\vec{r}) = \\
        &= \int_\Omega d^3r \hat{\psi}_\sigma^\dagger(\vec{r})\left(-\frac{\hbar^2}{2m} \nabla \cdot \nabla\right) \hat{\psi}_\sigma(\vec{r})
    \end{align*}

    Possiamo utilizzare l'integrazione per parti nel caso a più variabili, che sfrutta il teorema della divergenza:
    \begin{equation}\label{partial_int}
        \int_\Omega u \nabla \cdot \vec{v} = \int_\Gamma u \vec{v} \cdot \hat{n} - \int_\Omega \nabla u \cdot \vec{v}
    \end{equation}
    Dove $\Gamma$ è la superficie del volume $\Omega$. Siccome le particelle sono vincolate a rimanere nella scatola, la loro funzione d'onda
    deve annullarsi sulla superficie della scatola, quindi il contributo dell'integrale di superficie è nullo.

    \begin{align*}
        &\int_\Omega d^3r \hat{\psi}_\sigma^\dagger(\vec{r})\left(-\frac{\hbar^2}{2m} \nabla \cdot \nabla\right) \hat{\psi}_\sigma(\vec{r}) =\\
        &= \cancel{-\frac{\hbar^2}{2m} \int_\Gamma \hat{\psi}_\sigma^\dagger(\vec{r}) \nabla \hat{\psi}_\sigma(\vec{r})}
        + \frac{\hbar^2}{2m} \int_\Omega d^3r \nabla \hat{\psi}_\sigma^\dagger(\vec{r}) \cdot \nabla \hat{\psi}_\sigma(\vec{r})
    \end{align*}


    \begin{align*}
        & \left[\hat{T}, \hat{\psi}_\sigma^\dagger(\vec{r}, t)\right] = \\
        &= \left[\sum_{\sigma'} \frac{\hbar^2}{2m} \int_\Omega d^3r' \nabla \hat{\psi}_{\sigma'}^\dagger(\pvec{r}', t) \cdot \nabla \hat{\psi}_{\sigma'}(\pvec{r}', t), \hat{\psi}_\sigma^\dagger(\vec{r}, t)\right] = \\
        &= \frac{\hbar^2}{2m} \sum_{\sigma'} \int_\Omega d^3r' \left[\nabla \hat{\psi}_{\sigma'}^\dagger(\pvec{r}', t) \cdot \nabla \hat{\psi}_{\sigma'}(\pvec{r}', t), \hat{\psi}_\sigma^\dagger(\vec{r}, t)\right]  = \\
        &= \frac{\hbar^2}{2m} \sum_{\sigma'} \int_\Omega d^3r' \nabla \hat{\psi}_{\sigma'}^\dagger(\pvec{r}', t) \cdot \nabla \hat{\psi}_{\sigma'}(\pvec{r}', t)\hat{\psi}_\sigma^\dagger(\vec{r}, t)
        - \hat{\psi}_\sigma^\dagger(\vec{r}, t)\nabla \hat{\psi}_{\sigma'}^\dagger(\pvec{r}', t) \cdot \nabla \hat{\psi}_{\sigma'}(\pvec{r}', t)
        % \\
    \end{align*}
    Il gradiente agisce solo sulle coordinate $\pvec{r}'$, quindi posso portare $\hat{\psi}_\sigma^\dagger(\vec{r}, t)$ dentro il gradiente.
    \begin{align*}
        &\frac{\hbar^2}{2m} \sum_{\sigma'} \int_\Omega d^3r' \nabla \hat{\psi}_{\sigma'}^\dagger(\pvec{r}', t) \cdot \nabla \hat{\psi}_{\sigma'}(\pvec{r}', t)\hat{\psi}_\sigma^\dagger(\vec{r}, t)
        - \nabla \hat{\psi}_{\sigma'}^\dagger(\pvec{r}', t) \cdot \hat{\psi}_\sigma^\dagger(\vec{r}, t) \nabla \hat{\psi}_{\sigma'}(\pvec{r}', t)
        \\
        &\phantom{\frac{\hbar^2}{2m} \sum_{\sigma'} \int_\Omega d^3r'}+ \nabla \hat{\psi}_{\sigma'}^\dagger(\pvec{r}', t) \cdot \hat{\psi}_\sigma^\dagger(\vec{r}, t) \nabla \hat{\psi}_{\sigma'}(\pvec{r}', t)
        - \hat{\psi}_\sigma^\dagger(\vec{r}, t)\nabla \hat{\psi}_{\sigma'}^\dagger(\pvec{r}', t) \cdot \nabla \hat{\psi}_{\sigma'}(\pvec{r}', t) =
        \\
        &= \frac{\hbar^2}{2m} \sum_{\sigma'} \int_\Omega d^3r' \nabla \hat{\psi}_{\sigma'}^\dagger(\pvec{r}', t) \cdot \nabla\left( \hat{\psi}_{\sigma'}(\pvec{r}', t) \hat{\psi}_\sigma^\dagger(\vec{r}, t)
        - \hat{\psi}_\sigma^\dagger(\vec{r}, t) \hat{\psi}_{\sigma'}(\pvec{r}', t)\right)
        \\
        &\phantom{\frac{\hbar^2}{2m} \sum_{\sigma'} \int_\Omega d^3r'} + \nabla \hat{\psi}_{\sigma'}^\dagger(\pvec{r}', t) \cdot \hat{\psi}_\sigma^\dagger(\vec{r}, t) \nabla \hat{\psi}_{\sigma'}(\pvec{r}', t)
        - \hat{\psi}_\sigma^\dagger(\vec{r}, t)\nabla \hat{\psi}_{\sigma'}^\dagger(\pvec{r}', t) \cdot \nabla \hat{\psi}_{\sigma'}(\pvec{r}', t) =
        \\
        &= \frac{\hbar^2}{2m} \sum_{\sigma'} \int_\Omega d^3r' \nabla \hat{\psi}_{\sigma'}^\dagger(\pvec{r}', t) \cdot \nabla\left[ \hat{\psi}_{\sigma'}(\pvec{r}', t) \hat{\psi}_\sigma^\dagger(\vec{r}, t)\right]
        \\
        &\phantom{\frac{\hbar^2}{2m} \sum_{\sigma'} \int_\Omega d^3r'} + \nabla \hat{\psi}_{\sigma'}^\dagger(\pvec{r}', t)\hat{\psi}_\sigma^\dagger(\vec{r}, t) \cdot  \nabla \hat{\psi}_{\sigma'}(\pvec{r}', t)
        - \nabla \left(\hat{\psi}_\sigma^\dagger(\vec{r}, t) \hat{\psi}_{\sigma'}^\dagger(\pvec{r}', t)\right) \cdot \nabla \hat{\psi}_{\sigma'}(\pvec{r}', t) =
        \\
        &= \frac{\hbar^2}{2m} \sum_{\sigma'} \int_\Omega d^3r' \nabla \hat{\psi}_{\sigma'}^\dagger(\pvec{r}', t) \cdot \nabla\left[ \hat{\psi}_{\sigma'}(\pvec{r}', t), \hat{\psi}_\sigma^\dagger(\vec{r}, t)\right]
        + \nabla \left[ \hat{\psi}_{\sigma'}^\dagger(\pvec{r}', t), \hat{\psi}_\sigma^\dagger(\vec{r}, t)\right] \cdot  \nabla \hat{\psi}_{\sigma'}(\pvec{r}', t) =
    \end{align*}

    Nel  penultimo passaggio ho sfruttato il fatto che
    \begin{equation*}
        \nabla \hat{\psi}_{\sigma'}^\dagger(\pvec{r}', t) \cdot \hat{\psi}_\sigma^\dagger(\vec{r}, t) \nabla \hat{\psi}_{\sigma'}(\pvec{r}', t) =
        \nabla \hat{\psi}_{\sigma'}^\dagger(\pvec{r}', t) \hat{\psi}_\sigma^\dagger(\vec{r}, t) \cdot  \nabla \hat{\psi}_{\sigma'}(\pvec{r}', t) =
        \nabla \left(\hat{\psi}_{\sigma'}^\dagger(\pvec{r}', t) \hat{\psi}_\sigma^\dagger(\vec{r}, t)\right) \cdot  \nabla \hat{\psi}_{\sigma'}(\pvec{r}', t)
    \end{equation*}
    Ora possiamo sfruttare i due commutatori
    \begin{align}\label{psi_comm}
        \left[ \hat{\psi}_{\sigma'}(\pvec{r}', t), \hat{\psi}_\sigma^\dagger(\vec{r}, t)\right] = \delta_{\sigma \sigma'}\delta(\pvec{r}' - \vec{r})
        &&
        \left[ \hat{\psi}_{\sigma'}^\dagger(\pvec{r}', t), \hat{\psi}_\sigma^\dagger(\vec{r}, t)\right] = 0
    \end{align}

    \begin{align*}
        &\frac{\hbar^2}{2m} \sum_{\sigma'} \int_\Omega d^3r' \nabla \hat{\psi}_{\sigma'}^\dagger(\pvec{r}', t) \cdot \nabla \delta_{\sigma \sigma'}\delta(\pvec{r}' - \vec{r}) = \\
        &= \frac{\hbar^2}{2m} \sum_{\sigma'} \int_\Omega d^3r' \nabla \cdot \nabla \hat{\psi}_{\sigma'}^\dagger(\pvec{r}', t) \delta_{\sigma \sigma'}\delta(\pvec{r}' - \vec{r}) = \\
        &= \frac{\hbar^2}{2m} \sum_{\sigma'} \delta_{\sigma \sigma'} \int_\Omega d^3r' \nabla^2 \hat{\psi}_{\sigma'}^\dagger(\pvec{r}', t) \delta(\pvec{r}' - \vec{r}) = \\
        &= \frac{\hbar^2}{2m} \sum_{\sigma'} \delta_{\sigma \sigma'} \nabla^2 \hat{\psi}_{\sigma'}^\dagger(\vec{r}, t) = \\
        &= \frac{\hbar^2}{2m} \nabla^2 \hat{\psi}_\sigma^\dagger(\vec{r}, t)
    \end{align*}

    Nel secondo passaggio si è usata ancora \eqref{partial_int} (ma al contrario), trascurando sempre l'integrale di superficie.

    Per il termine di interazione abbiamo
    \begin{align*}
        &\left[\hat{V}, \hat{\psi}_\sigma^\dagger(\vec{r}, t)\right]  =
        \left[\frac{1}{2} \sum_{\sigma', \sigma''} \int_\Omega d^3r' d^3r'' v(\pvec{r}' - \pvec{r}'') \hat{\psi}_{\sigma'}^\dagger(\pvec{r}')\hat{\psi}_{\sigma''}^\dagger(\pvec{r}'')\hat{\psi}_{\sigma''}(\pvec{r}'')\hat{\psi}_{\sigma'}(\pvec{r}'), \hat{\psi}_\sigma^\dagger(\vec{r}, t)\right] =
        \\
        &= \frac{1}{2} \sum_{\sigma', \sigma''} \int_\Omega d^3r' d^3r'' \left[v(\pvec{r}' - \pvec{r}'') \hat{\psi}_{\sigma'}^\dagger(\pvec{r}')\hat{\psi}_{\sigma''}^\dagger(\pvec{r}'')\hat{\psi}_{\sigma''}(\pvec{r}'')\hat{\psi}_{\sigma'}(\pvec{r}'), \hat{\psi}_\sigma^\dagger(\vec{r}, t)\right]
    \end{align*}

    Grazie all'identità deil commutatori
    \begin{equation}\label{comm_id}
        \left[AB, C\right] = A \left[B, C\right] + \left[A, C\right] B
    \end{equation}
    troviamo
    \begin{align*}
        &\frac{1}{2} \sum_{\sigma', \sigma''} \int_\Omega d^3r' d^3r'' v(\pvec{r}' - \pvec{r}'') \hat{\psi}_{\sigma'}^\dagger(\pvec{r}')\hat{\psi}_{\sigma''}^\dagger(\pvec{r}'')\left[\hat{\psi}_{\sigma''}(\pvec{r}'')\hat{\psi}_{\sigma'}(\pvec{r}'), \hat{\psi}_\sigma^\dagger(\vec{r}, t)\right] \\
        &\phantom{\frac{1}{2} \sum_{\sigma', \sigma''} \int_\Omega} + \left[v(\pvec{r}' - \pvec{r}'') \hat{\psi}_{\sigma'}^\dagger(\pvec{r}')\hat{\psi}_{\sigma''}^\dagger(\pvec{r}''), \hat{\psi}_\sigma^\dagger(\vec{r}, t)\right] \hat{\psi}_{\sigma''}(\pvec{r}'')\hat{\psi}_{\sigma'}(\pvec{r}') =
        \\
        &= \frac{1}{2} \sum_{\sigma', \sigma''} \int_\Omega d^3r' d^3r'' v(\pvec{r}' - \pvec{r}'') \hat{\psi}_{\sigma'}^\dagger(\pvec{r}')\hat{\psi}_{\sigma''}^\dagger(\pvec{r}'')\left[\hat{\psi}_{\sigma''}(\pvec{r}'')\hat{\psi}_{\sigma'}(\pvec{r}'), \hat{\psi}_\sigma^\dagger(\vec{r}, t)\right] \\
        &\phantom{\frac{1}{2} \sum_{\sigma', \sigma''} \int_\Omega} + \left[v(\pvec{r}' - \pvec{r}'') \hat{\psi}_{\sigma'}^\dagger(\vec{r}')\hat{\psi}_{\sigma''}^\dagger(\pvec{r}''), \hat{\psi}_{\sigma'}^\dagger(\vec{r}, t)\right] \hat{\psi}_{\sigma''}(\pvec{r}'')\hat{\psi}_{\sigma'}(\pvec{r}') =
        \\
        &= \frac{1}{2} \sum_{\sigma', \sigma''} \int_\Omega d^3r' d^3r'' v(\pvec{r}' - \pvec{r}'') \hat{\psi}_{\sigma'}^\dagger(\pvec{r}')\hat{\psi}_{\sigma''}^\dagger(\pvec{r}'')\left[\hat{\psi}_{\sigma''}(\pvec{r}'')\hat{\psi}_{\sigma'}(\pvec{r}'), \hat{\psi}_\sigma^\dagger(\vec{r}, t)\right] \\
        &\phantom{\frac{1}{2} \sum_{\sigma', \sigma''} \int_\Omega} + v(\pvec{r}' - \pvec{r}'') \left[\hat{\psi}_{\sigma'}^\dagger(\pvec{r}')\hat{\psi}_{\sigma''}^\dagger(\pvec{r}''), \hat{\psi}_\sigma^\dagger(\vec{r}, t)\right] \hat{\psi}_{\sigma''}(\pvec{r}'')\hat{\psi}_{\sigma'}(\pvec{r}')
    \end{align*}
Applicando nuovamente \eqref{comm_id} si nota che dal secondo commutatore si ottengono due commutatori che danno entrambi 0.
Applichiamo \eqref{comm_id} anche al primo, trovando
\begin{align*}
    &\frac{1}{2} \sum_{\sigma', \sigma''} \int_\Omega d^3r' d^3r'' v(\pvec{r}' - \pvec{r}'') \hat{\psi}_{\sigma'}^\dagger(\pvec{r}')\hat{\psi}_{\sigma''}^\dagger(\pvec{r}'')\left[\hat{\psi}_{\sigma''}(\pvec{r}'')\hat{\psi}_{\sigma'}(\pvec{r}'), \hat{\psi}_\sigma^\dagger(\vec{r}, t)\right]
    \\
    &= \frac{1}{2} \sum_{\sigma', \sigma''} \int_\Omega d^3r' d^3r'' v(\pvec{r}' - \pvec{r}'') \hat{\psi}_{\sigma'}^\dagger(\pvec{r}')\hat{\psi}_{\sigma''}^\dagger(\pvec{r}'') \left(
        \hat{\psi}_{\sigma''}(\pvec{r}'') \left[\hat{\psi}_{\sigma'}(\pvec{r}'), \hat{\psi}_\sigma^\dagger(\vec{r}, t)\right]
        + \left[\hat{\psi}_{\sigma''}(\pvec{r}''), \hat{\psi}_\sigma^\dagger(\vec{r}, t)\right] \hat{\psi}_{\sigma'}(\pvec{r}')
    \right)
    \\
    &= \frac{1}{2} \sum_{\sigma', \sigma''} \int_\Omega d^3r' d^3r'' v(\pvec{r}' - \pvec{r}'') \hat{\psi}_{\sigma'}^\dagger(\pvec{r}')\hat{\psi}_{\sigma''}^\dagger(\pvec{r}'') \left(
        \hat{\psi}_{\sigma''}(\pvec{r}'') \delta_{\sigma' \sigma}\delta(\pvec{r}' - \vec{r})
        + \delta_{\sigma'' \sigma}\delta(\pvec{r}'' - \vec{r}) \hat{\psi}_{\sigma'}(\pvec{r}')
    \right) =
    \\
    &= \frac{1}{2} \sum_{\sigma', \sigma''} \int_\Omega d^3r' d^3r'' v(\pvec{r}' - \pvec{r}'') \hat{\psi}_{\sigma'}^\dagger(\pvec{r}')\hat{\psi}_{\sigma''}^\dagger(\pvec{r}'')
        \hat{\psi}_{\sigma''}(\pvec{r}'') \delta_{\sigma' \sigma}\delta(\pvec{r}' - \vec{r}) \\
    &\phantom{=}+\frac{1}{2} \sum_{\sigma', \sigma''} \int_\Omega d^3r' d^3r'' v(\pvec{r}' - \pvec{r}'') \hat{\psi}_{\sigma'}^\dagger(\pvec{r}')\hat{\psi}_{\sigma''}^\dagger(\pvec{r}'') \delta_{\sigma'' \sigma}\delta(\pvec{r}'' - \vec{r}) \hat{\psi}_{\sigma'}(\pvec{r}') =
    \\
    &= \frac{1}{2} \sum_{\sigma''} \int_\Omega d^3r'' v(\vec{r} - \pvec{r}'') \hat{\psi}_{\sigma}^\dagger(\vec{r})\hat{\psi}_{\sigma''}^\dagger(\pvec{r}'')
        \hat{\psi}_{\sigma''}(\pvec{r}'')
    +\frac{1}{2} \sum_{\sigma'} \int_\Omega d^3r' v(\pvec{r}' - \vec{r}) \hat{\psi}_{\sigma'}^\dagger(\pvec{r}')\hat{\psi}_{\sigma}^\dagger(\vec{r}) \hat{\psi}_{\sigma'}(\pvec{r}') =
    \\
    &= \frac{1}{2} \sum_{\sigma'} \int_\Omega d^3r' v(\vec{r} - \pvec{r}') \hat{\psi}_{\sigma}^\dagger(\vec{r})\hat{\psi}_{\sigma'}^\dagger(\pvec{r}')
        \hat{\psi}_{\sigma'}(\pvec{r}')
    +\frac{1}{2} \sum_{\sigma'} \int_\Omega d^3r' v(\pvec{r}' - \vec{r}) \hat{\psi}_{\sigma'}^\dagger(\pvec{r}')\hat{\psi}_{\sigma}^\dagger(\vec{r}) \hat{\psi}_{\sigma'}(\pvec{r}') =
\end{align*}

    Sfruttando $[\hat{\psi}_{\sigma}^\dagger(\vec{r}), \hat{\psi}_{\sigma'}^\dagger(\pvec{r}')] = 0$ per scambiare i due operatori centrali  e $v(\vec{r} - \pvec{r}') = v(\pvec{r}' - \vec{r})$ possiamo sommare i due termini.
    Infine otteniamo

    \begin{align*}
        \left[\hat{V},\hat{\psi}_\sigma^\dagger(\vec{r}, t) \right] = \sum_{\sigma'} \int_\Omega d^3r' v(\vec{r} - \pvec{r}') \hat{\psi}_{\sigma}^\dagger(\vec{r})\hat{\psi}_{\sigma'}^\dagger(\pvec{r}')
        \hat{\psi}_{\sigma'}(\pvec{r}')
    \end{align*}

    Per $\hat{\psi}_\sigma(\vec{r}, t)$ troviamo
    \begin{equation*}
        \partial_t \hat{\psi}_\sigma(\vec{r}, t) = \left(\partial_t \hat{\psi}_\sigma^\dagger(\vec{r}, t)\right)^\dagger =
        -\frac{i}{\hbar}\left[\hat{H}, \hat{\psi}_\sigma^\dagger(\vec{r}, t) \right]^\dagger
    \end{equation*}

    \begin{align*}
        -\frac{i}{\hbar}\left[\hat{T}, \hat{\psi}_\sigma^\dagger(\vec{r}, t) \right]^\dagger = -\frac{i}{\hbar}\left(\frac{\hbar^2}{2m} \nabla^2 \hat{\psi}_\sigma(\vec{r}, t)\right) =
        -\frac{i\hbar}{2m} \nabla^2 \hat{\psi}_\sigma(\vec{r}, t)
    \end{align*}

    \begin{align*}
        &-\frac{i}{\hbar}\left[\hat{V}, \hat{\psi}_\sigma^\dagger(\vec{r}, t) \right]^\dagger =
        \\
        &=  -\frac{i}{\hbar} \sum_{\sigma'} \int_\Omega d^3r' v(\vec{r} - \pvec{r}') \left(\hat{\psi}_{\sigma}^\dagger(\vec{r})\hat{\psi}_{\sigma'}^\dagger(\pvec{r}')
        \hat{\psi}_{\sigma'}(\pvec{r}')\right)^\dagger =
        \\
        &=  -\frac{i}{\hbar} \sum_{\sigma'} \int_\Omega d^3r' v(\vec{r} - \pvec{r}') \hat{\psi}_{\sigma'}^\dagger(\pvec{r}') \hat{\psi}_{\sigma'}(\pvec{r}')
        \hat{\psi}_{\sigma}(\vec{r})
    \end{align*}

    \begin{align*}
        &\partial_t \hat{n} = \hat{\psi}_\sigma^\dagger(\vec{r}, t) \partial_t \hat{\psi}_\sigma(\vec{r}, t) + \partial_t \hat{\psi}_\sigma^\dagger(\vec{r}, t) \hat{\psi}_\sigma(\vec{r}, t) =
        \\
        &= \hat{\psi}_\sigma^\dagger(\vec{r}, t) \left(-\frac{i\hbar}{2m} \nabla^2 \hat{\psi}_\sigma(\vec{r}, t)\right) +
        \left(\frac{i\hbar}{2m} \nabla^2 \hat{\psi}_\sigma^\dagger(\vec{r}, t)\right) \hat{\psi}_\sigma(\vec{r}, t)
        \\
        &= -\frac{i}{\hbar} \sum_{\sigma'} \int_\Omega d^3r' v(\vec{r} - \pvec{r}') \hat{\psi}_\sigma^\dagger(\vec{r}, t) \hat{\psi}_{\sigma'}^\dagger(\pvec{r}') \hat{\psi}_{\sigma'}(\pvec{r}')
        \hat{\psi}_{\sigma}(\vec{r})
        \\
        &= +\frac{i}{\hbar} \sum_{\sigma'} \int_\Omega d^3r' v(\vec{r} - \pvec{r}') \hat{\psi}_{\sigma}^\dagger(\vec{r})\hat{\psi}_{\sigma'}^\dagger(\pvec{r}')
        \hat{\psi}_{\sigma'}(\pvec{r}') \hat{\psi}_\sigma(\vec{r}, t) =
        \\
        &= \frac{i\hbar}{2m} \left[\nabla^2 \hat{\psi}_\sigma^\dagger(\vec{r}, t)\hat{\psi}_\sigma(\vec{r}, t)
        - \hat{\psi}_\sigma^\dagger(\vec{r}, t) \nabla^2 \hat{\psi}_\sigma(\vec{r}, t) \right] =
        \\
        &= \frac{i\hbar}{2m} \left[\nabla^2 \hat{\psi}_\sigma^\dagger(\vec{r}, t)\hat{\psi}_\sigma(\vec{r}, t)
        - \hat{\psi}_\sigma^\dagger(\vec{r}, t) \nabla^2 \hat{\psi}_\sigma(\vec{r}, t)
        + \nabla \hat{\psi}_\sigma^\dagger(\vec{r}, t) \cdot \nabla \hat{\psi}_\sigma(\vec{r}, t)
        - \nabla \hat{\psi}_\sigma^\dagger(\vec{r}, t) \cdot \nabla \hat{\psi}_\sigma(\vec{r}, t)
        \right] =
        \\
        &= \frac{i\hbar}{2m} \nabla \cdot \left(\nabla \hat{\psi}_\sigma^\dagger(\vec{r}, t)\hat{\psi}_\sigma(\vec{r}, t)
        - \hat{\psi}_\sigma^\dagger(\vec{r}, t) \nabla\hat{\psi}_\sigma(\vec{r}, t) \right)
        \\
        &= -\frac{i\hbar}{2im} \nabla \cdot \left(\hat{\psi}_\sigma^\dagger(\vec{r}, t) \nabla\hat{\psi}_\sigma(\vec{r}, t)
         - \nabla \hat{\psi}_\sigma^\dagger(\vec{r}, t)\hat{\psi}_\sigma(\vec{r}, t)\right)
    \end{align*}

    Definendo
    \begin{equation*}
        \hat{j} \equiv \frac{\hbar}{2im} \left( \hat{\psi}_\sigma^\dagger(\vec{r}, t) \nabla\hat{\psi}_\sigma(\vec{r}, t)
        - \nabla \hat{\psi}_\sigma^\dagger(\vec{r}, t)\hat{\psi}_\sigma(\vec{r}, t) \right)
    \end{equation*}

    Troviamo
        \begin{equation*}
            \partial_t \hat{n} = - \nabla \cdot \hat{j}
        \end{equation*}
    \part

    Per $\vec{k} = \pvec{k}'$
    \begin{align*}
        \int_\Omega d^3r \exp\left(i 0 \cdot \vec{r}\right) = \int_\Omega d^3r = \Omega
    \end{align*}

    Per $\vec{k} \neq \pvec{k}'$

    \begin{align*}
        &\int_\Omega d^3r \exp\left(i(\vec{k} - \pvec{k}') \cdot \vec{r}\right) \\
        &= \int_{-L/2}^{L/2}\int_{-L/2}^{L/2}\int_{-L/2}^{L/2} dx dy dz \exp \left(i (k_x - k'_x)x + i (k_y - k'_y)y + i (k_z - k'_z)z \right)\\
        &= \left(\int_{-L/2}^{L/2}dx \exp \left(i (k_x - k'_x)x \right)\right)^3
    \end{align*}

    \begin{align*}
        &\int_{-L/2}^{L/2}dx \exp \left(i (k_x - k'_x)x \right) = \left[\frac{\exp(i(k_x - k'_x)x)}{i(k_x - k'_x)}\right]_{-L/2}^{L/2} = \\
        &= \frac{\exp(i(k_x - k'_x)\frac{L}{2)} - \exp(-i(k_x - k'_x)\frac{L}{2})}{i(k_x - k'_x)}
    \end{align*}

    $k_x - k'_x = \frac{2 \pi}{L} (n_x - n'_x) = \frac{2 \pi}{L} \Delta n, \ \Delta n \in \mathbb{Z} - \left\{0\right\}$

    \begin{align*}
        &\frac{\exp(i\frac{2 \pi}{L} \Delta n \frac{L}{2}) - \exp(-i\frac{2 \pi}{L} \Delta n \frac{L}{2})}{i\frac{2 \pi}{L} \Delta n} = \\
        &= \frac{L}{\pi \Delta n}\frac{\exp(i\pi\Delta n) - \exp(-i\pi\Delta n)}{2i} = \\
        &= L\frac{\sin(\pi \Delta n)}{\pi \Delta n} = 0
    \end{align*}

    \part

    \begin{align*}
        &\sum_\sigma \int_\Omega d^3r \hat{\psi}_\sigma^\dagger(\vec{r})\hat{\psi}_\sigma(\vec{r}) = \\
        &= \frac{1}{\Omega}\sum_\sigma \int_\Omega d^3r \left(\sum_{\pvec{k}'}\exp\left(-i \pvec{k}'\cdot\vec{r}\right)\hat{c}_{\pvec{k}', \sigma}^\dagger\right)\left(\sum_{\vec{k}}\exp\left(i \vec{k}\cdot\vec{r}\right)\hat{c}_{\vec{k}, \sigma}\right) = \\
        &= \frac{1}{\Omega}\sum_\sigma \int_\Omega d^3r \sum_{\vec{k}, \pvec{k}'}\exp\left(i \left(\vec{k} - \pvec{k}'\right)\cdot\vec{r}\right)\hat{c}_{\pvec{k}', \sigma}^\dagger\hat{c}_{\vec{k}, \sigma} = \\
        &= \frac{1}{\Omega}\sum_\sigma \sum_{\vec{k}, \pvec{k}'}\hat{c}_{\pvec{k}', \sigma}^\dagger\hat{c}_{\vec{k}, \sigma} \int_\Omega d^3r \exp\left(i \left(\vec{k} - \pvec{k}'\right)\cdot\vec{r}\right) = \\
        &= \frac{1}{\Omega}\sum_\sigma \sum_{\vec{k}, \pvec{k}'}\hat{c}_{\pvec{k}', \sigma}^\dagger\hat{c}_{\vec{k}, \sigma} \Omega \delta_{\vec{k}, \pvec{k}'} = \sum_{\vec{k}, \sigma} \hat{c}_{\vec{k}, \sigma}^\dagger\hat{c}_{\vec{k}, \sigma}\\
    \end{align*}

    \part

    \begin{align*}
        \hat{c}_{\vec{k}, \sigma} = \frac{1}{\sqrt{\Omega}} \int_\Omega d^3r \exp(-i \vec{k}\cdot\vec{r}) \hat{\psi}_\sigma(\vec{r})
        &&
        \hat{c}_{\vec{k}, \sigma}^\dagger = \frac{1}{\sqrt{\Omega}} \int_\Omega d^3r \exp(i \vec{k}\cdot\vec{r}) \hat{\psi}_\sigma^\dagger(\vec{r})
    \end{align*}

    \begin{align*}
        &\left[\hat{c}_{\vec{k}, \sigma}, \hat{c}_{\pvec{k}, \sigma'}^\dagger\right] = \\
        &= \frac{1}{\Omega}\left[\int_\Omega d^3r \exp(-i \vec{k}\cdot\vec{r}) \hat{\psi}_\sigma(\vec{r}),
                    \int_\Omega d^3r' \exp(i \pvec{k}'\cdot\pvec{r}') \hat{\psi}_{\sigma'}^\dagger(\pvec{r}')\right] =
        \\
        &= \frac{1}{\Omega}\int_\Omega d^3r d^3r' \exp\left(i \pvec{k}'\cdot\pvec{r}' - i\vec{k}\cdot\vec{r}\right)\left[\hat{\psi}_\sigma(\vec{r}),
        \hat{\psi}_{\sigma'}^\dagger(\pvec{r}')\right] =
        \\
        &= \frac{1}{\Omega}\int_\Omega d^3r d^3r' \exp\left(i \pvec{k}'\cdot\pvec{r}' - i\vec{k}\cdot\vec{r}\right)\delta_{\sigma \sigma'}\delta(\vec{r} - \pvec{r}') = \\
        &= \frac{\delta_{\sigma \sigma'}}{\Omega}\int_\Omega d^3r d^3r' \exp\left(i \pvec{k}'\cdot\pvec{r} - i\vec{k}\cdot\vec{r}\right) = \frac{\delta_{\sigma \sigma'}}{\Omega} \Omega \delta_{\vec{k}, \pvec{k}'} =
        \delta_{\sigma \sigma'} \delta_{\vec{k}, \pvec{k}'}
    \end{align*}

    \part

    \begin{align*}
        &\Braket{SF | SF} = \Braket{ 0 | \left(\prod_{\vec{k} \leq k_F} \hat{c}_{\vec{k}, \uparrow}^\dagger \hat{c}_{\vec{k}, \downarrow}^\dagger \right)^\dagger \prod_{\vec{k} \leq k_F} \hat{c}_{\vec{k}, \uparrow}^\dagger \hat{c}_{\vec{k}, \downarrow}^\dagger | 0} = \\
        &= \Braket{ 0 | \prod_{\vec{k} \leq k_F} \hat{c}_{\vec{k}, \downarrow} \hat{c}_{\vec{k}, \uparrow} \hat{c}_{\vec{k}, \uparrow}^\dagger \hat{c}_{\vec{k}, \downarrow}^\dagger | 0} = \\
    \end{align*}

    Possiamo usare la formula \eqref{psi_comm}, e siccome $\vec{k} = \pvec{k}'$ e $\sigma = \sigma'$ troviamo
    \begin{equation*}
        \hat{c}_{\vec{k}, \uparrow} \hat{c}_{\vec{k}, \uparrow}^\dagger = 1 + \hat{c}_{\vec{k}, \uparrow}^\dagger \hat{c}_{\vec{k}, \uparrow}
    \end{equation*}
    Quindi
    \begin{align*}
        &\Braket{ 0 | \prod_{\vec{k} \leq k_F} \hat{c}_{\vec{k}, \downarrow} \hat{c}_{\vec{k}, \downarrow}^\dagger + \hat{c}_{\vec{k}, \downarrow} \hat{c}_{\vec{k}, \uparrow}^\dagger \hat{c}_{\vec{k}, \uparrow} \hat{c}_{\vec{k}, \downarrow}^\dagger | 0}
    \end{align*}

    Dato che  nel secondo termine $\hat{c}_{\vec{k}, \uparrow}$ viene applicato a $\Ket{0}$ prima di $\hat{c}_{\vec{k}, \uparrow}^\dagger$, significa che stiamo cercando di distruggere una particella con vettore d'onda $\vec{k}$ e spin $\uparrow$ prima che questa venga creata,
    quindi questo termine applicato a $\Ket{0}$ darà $0$ e quindi possiamo ignorarlo.

    \begin{align*}
        &\Braket{ 0 | \prod_{\vec{k} \leq k_F} \hat{c}_{\vec{k}, \downarrow} \hat{c}_{\vec{k}, \downarrow}^\dagger | 0} = \\
        &= \Braket{ 0 | \prod_{\vec{k} \leq k_F} 1 + \hat{c}_{\vec{k}, \downarrow}^\dagger \hat{c}_{\vec{k}, \downarrow} | 0}
    \end{align*}
    Qui possiamo riutilizzare lo stesso argomento precedente e troviamo
    \begin{align*}
        \Braket{ 0 | \prod_{\vec{k} \leq k_F} 1 | 0} = \Braket{0 | 0} = 1
    \end{align*}

    \part

    Nei calcoli successivi non è riportato il termine di normalizzazione $1/\sqrt{\Omega}$, ma dato che si semplificherebbero come nell'esercizio di prima
    possiamo ometterli senza problemi.
    \begin{align*}
        &\hat{T} = \sum_\sigma \int_\Omega d^3r \hat{\psi}_\sigma^\dagger(\vec{r})\left(-\frac{\hbar^2}{2m} \nabla^2\right) \hat{\psi}_\sigma(\vec{r}) =
        \\
        &= \sum_\sigma \int_\Omega d^3r \left(\sum_{\pvec{k}'}\exp\left(-i \pvec{k}'\cdot\vec{r}\right)\hat{c}_{\pvec{k}', \sigma}^\dagger\right) \left(-\frac{\hbar^2}{2m} \nabla^2\right) \left(\sum_{\vec{k}}\exp\left(i \vec{k}\cdot\vec{r}\right)\hat{c}_{\vec{k}, \sigma}\right) =
        \\
        &= -\frac{\hbar^2}{2m} \sum_\sigma \sum_{\vec{k}, \pvec{k}'} \hat{c}_{\pvec{k}', \sigma}^\dagger \hat{c}_{\vec{k}, \sigma} \int_\Omega d^3r \exp\left(-i \pvec{k}'\cdot\vec{r}\right)  \nabla^2 \exp\left(i \vec{k}\cdot\vec{r}\right) =
        \\
        &= -\frac{\hbar^2}{2m} \sum_\sigma \sum_{\vec{k}, \pvec{k}'} \hat{c}_{\pvec{k}', \sigma}^\dagger \hat{c}_{\vec{k}, \sigma} \int_\Omega d^3r \exp\left(-i \pvec{k}'\cdot\vec{r}\right)  \nabla^2 \exp\left(i \vec{k}\cdot\vec{r}\right)
    \end{align*}

    Per il laplaciano vale
    \begin{equation*}
        \nabla^2 \exp\left(i \vec{k}\cdot\vec{r}\right)  =
        \left((i k_x)^2 \exp\left(i \vec{k}\cdot\vec{r}\right),\,
        (i k_y)^2 \exp\left(i \vec{k}\cdot\vec{r}\right),\,
        (i k_y)^2 \exp\left(i \vec{k}\cdot\vec{r}\right)
        \right) = -\pvec{k}^2 \exp\left(i \vec{k}\cdot\vec{r}\right)
    \end{equation*}

    Quindi troviamo
    \begin{align*}
        &\frac{\hbar^2}{2m} \sum_\sigma \sum_{\vec{k}, \pvec{k}'} \pvec{k}^2 \hat{c}_{\pvec{k}', \sigma}^\dagger  \hat{c}_{\vec{k}, \sigma} \int_\Omega d^3r \exp\left(-i \pvec{k}'\cdot\vec{r}\right) \exp\left(i \vec{k}\cdot\vec{r}\right) 
        \\
        &= \frac{\hbar^2}{2m} \sum_\sigma \sum_{\vec{k}, \pvec{k}'} \pvec{k}^2 \hat{c}_{\pvec{k}', \sigma}^\dagger  \hat{c}_{\vec{k}, \sigma} \delta_{\vec{k}, \pvec{k}'}
        = \frac{\hbar^2}{2m} \sum_\sigma \sum_{\vec{k}} \pvec{k}^2 \hat{c}_{\vec{k}, \sigma}^\dagger  \hat{c}_{\vec{k}, \sigma}
    \end{align*}

    \begin{align*}
        &\hat{V} = \frac{1}{2} \sum_{\sigma, \sigma'} \int_\Omega d^3r d^3r' v(\vec{r} - \pvec{r}') \hat{\psi}_\sigma^\dagger(\vec{r})\hat{\psi}_{\sigma'}^\dagger(\pvec{r}')\hat{\psi}_{\sigma'}(\pvec{r}')\hat{\psi}_\sigma(\vec{r}) = \\
        &=\frac{1}{2} \sum_{\sigma, \sigma'} \int_\Omega d^3r d^3r'
        \sum_{\vec{q}} v_{\vec{q}} \exp\left(i \vec{q} \cdot \left(\vec{r} - \pvec{r}'\right)\right)
        \left(\sum_{\pvec{k}_1}\exp\left(-i \pvec{k}_1\cdot\vec{r}\right)\hat{c}_{\pvec{k}_1, \sigma}^\dagger\right)
        \left(\sum_{\pvec{k}_2}\exp\left(-i \pvec{k}_2\cdot\pvec{r}'\right)\hat{c}_{\pvec{k}_2, \sigma'}^\dagger\right)\\
        &\cdot \left(\sum_{\pvec{k}_3}\exp\left(i \pvec{k}_3\cdot\pvec{r}'\right)\hat{c}_{\pvec{k}_3, \sigma'}\right)
        \left(\sum_{\pvec{k}_4}\exp\left(i \pvec{k}_4\cdot\vec{r}\right)\hat{c}_{\pvec{k}_4, \sigma}\right) =
        \\
        &= \frac{1}{2} \sum_{\sigma, \sigma'} \sum_{\vec{k}_{1,2,3,4}}
        v_{\vec{q}}\,\hat{c}_{\pvec{k}_1, \sigma}^\dagger \hat{c}_{\pvec{k}_2, \sigma'}^\dagger \hat{c}_{\pvec{k}_3, \sigma'}\hat{c}_{\pvec{k}_4, \sigma} 
        \\
        &\qquad\int_\Omega d^3r d^3r' \exp \left(i \vec{q} \cdot \left(\vec{r} - \pvec{r}'\right)\right) \exp\left(-i \pvec{k}_1\cdot\vec{r}\right)\exp\left(-i \pvec{k}_2\cdot\pvec{r}'\right)\exp\left(i \pvec{k}_3\cdot\pvec{r}'\right)\exp\left(i \pvec{k}_4\cdot\vec{r}\right) =
        \\
        &= \frac{1}{2} \sum_{\sigma, \sigma'} \sum_{\vec{k}_{1,2,3,4}}
        v_{\vec{q}}\,\hat{c}_{\pvec{k}_1, \sigma}^\dagger \hat{c}_{\pvec{k}_2, \sigma'}^\dagger \hat{c}_{\pvec{k}_3, \sigma'}\hat{c}_{\pvec{k}_4, \sigma}
        \int_\Omega d^3r d^3r' \exp\left(i \left(\vec{q} - \vec{k}_1 + \vec{k}_4\right)\cdot\vec{r}\right) \exp \left(i \left(\vec{k}_3 - \vec{k}_2 - \vec{q}\right)\cdot\pvec{r}'\right) =
        \\
        &= \frac{1}{2} \sum_{\sigma, \sigma'} \sum_{\vec{k}_{1,2,3,4}}
        v_{\vec{q}}\,\hat{c}_{\pvec{k}_1, \sigma}^\dagger \hat{c}_{\pvec{k}_2, \sigma'}^\dagger \hat{c}_{\pvec{k}_3, \sigma'}\hat{c}_{\pvec{k}_4, \sigma}
        \int_\Omega d^3r d^3r' \exp\left(i \left(\vec{k}_4 - \left(\vec{k}_1 - \vec{q}\right)\right)\cdot\vec{r}\right)
        \exp \left(i \left(\vec{k}_3 - \left(\vec{k}_2 + \vec{q}\right)\right)\cdot\pvec{r}'\right) =
        \\
        &= \frac{1}{2} \sum_{\sigma, \sigma'} \sum_{\vec{k}_{1,2,3,4}}
        v_{\vec{q}}\,\hat{c}_{\pvec{k}_1, \sigma}^\dagger \hat{c}_{\pvec{k}_2, \sigma'}^\dagger \hat{c}_{\pvec{k}_3, \sigma'}\hat{c}_{\pvec{k}_4, \sigma}
        \delta_{\vec{k}_4, \vec{k}_1 - \vec{q}}\,\delta_{\vec{k}_3, \vec{k}_2 + \vec{q}} =
        \\
        &= \frac{1}{2} \sum_{\sigma, \sigma'} \sum_{\vec{k}_{1,2}}
        v_{\vec{q}}\,\hat{c}_{\pvec{k}_1, \sigma}^\dagger \hat{c}_{\pvec{k}_2, \sigma'}^\dagger \hat{c}_{\pvec{k}_2 + \vec{q}, \sigma'}\hat{c}_{\pvec{k}_1 - \vec{q}, \sigma}
        = \frac{1}{2} \sum_{\sigma, \sigma'} \sum_{\vec{k}, \pvec{k}'}
        v_{\vec{q}}\,\hat{c}_{\vec{k}, \sigma}^\dagger \hat{c}_{\pvec{k}' \sigma'}^\dagger \hat{c}_{\pvec{k}' + \vec{q}, \sigma'}\hat{c}_{\vec{k} - \vec{q}, \sigma}
    \end{align*}

    Infine l'Hamiltoniana risulta
    \begin{equation*}
        \hat{H} = \frac{\hbar^2}{2m} \sum_\sigma \sum_{\vec{k}} \pvec{k}^2 \hat{c}_{\vec{k}, \sigma}^\dagger  \hat{c}_{\vec{k}, \sigma}
        +\frac{1}{2} \sum_{\sigma, \sigma'} \sum_{\vec{k}, \pvec{k}'}
        v_{\vec{q}}\,\hat{c}_{\vec{k}, \sigma}^\dagger \hat{c}_{\pvec{k}' \sigma'}^\dagger \hat{c}_{\pvec{k}' + \vec{q}, \sigma'}\hat{c}_{\vec{k} - \vec{q}, \sigma}
    \end{equation*}

\end{homeworkProblem}
\end{document}
