\documentclass{article}

\usepackage{fancyhdr}
\usepackage{extramarks}
\usepackage{amsmath}
\usepackage{amsthm}
\usepackage{amsfonts}
\usepackage{tikz}
\usepackage[plain]{algorithm}
\usepackage{algpseudocode}
\usepackage{microtype}

\usepackage{dsfont}

\usetikzlibrary{automata,positioning}

%
% Basic Document Settings
%

\topmargin=-0.45in
\evensidemargin=0in
\oddsidemargin=0in
\textwidth=6.5in
\textheight=9.0in
\headsep=0.25in

\linespread{1.1}

\pagestyle{fancy}
\lhead{\hmwkAuthorName}
\chead{\hmwkClass: \hmwkTitle}
\rhead{\firstxmark}
\lfoot{\lastxmark}
\cfoot{\thepage}

\renewcommand\headrulewidth{0.4pt}
\renewcommand\footrulewidth{0.4pt}

\setlength\parindent{0pt}

%
% Create Problem Sections
%

\newcommand{\enterProblemHeader}[1]{
    \nobreak\extramarks{Problema \arabic{#1}}{}
}

\newcommand{\exitProblemHeader}[1]{
    \nobreak\extramarks{Problema \arabic{#1}}{Problema \arabic{#1}}
    \stepcounter{#1}
    \nobreak\extramarks{Problema \arabic{#1}}{}
}

\setcounter{secnumdepth}{0}
\newcounter{partCounter}
\newcounter{homeworkProblemCounter}
\setcounter{homeworkProblemCounter}{1}
\nobreak\extramarks{Problema \arabic{homeworkProblemCounter}}{}\nobreak{}

%
% Homework Problem Environment
%
% This environment takes an optional argument. When given, it will adjust the
% problem counter. This is useful for when the problems given for your
% assignment aren't sequential. See the last 3 problems of this template for an
% example.
%
\newenvironment{homeworkProblem}[1][-1]{
    \ifnum#1>0
        \setcounter{homeworkProblemCounter}{#1}
    \fi
    \section{Problema \arabic{homeworkProblemCounter}}
    \setcounter{partCounter}{1}
    \enterProblemHeader{homeworkProblemCounter}
}{
    \exitProblemHeader{homeworkProblemCounter}
}

\newcommand{\hmwkTitle}{Compiti \#1}
\newcommand{\hmwkClass}{Fisica Teorica della Materia}
\newcommand{\hmwkAuthorName}{\textbf{Gianluca Zappavigna}}

\renewcommand{\part}[1]{\textbf{Parte \arabic{partCounter}}\stepcounter{partCounter}\\}

%
% Various Helper Commands
%

% Useful for algorithms
\newcommand{\alg}[1]{\textsc{\bfseries \footnotesize #1}}

% For derivatives
\newcommand{\deriv}[1]{\frac{\mathrm{d}}{\mathrm{d}x} (#1)}

% For partial derivatives
\newcommand{\pderiv}[2]{\frac{\partial}{\partial #1} #2}

% Integral dx
\newcommand{\dx}{\mathrm{d}x}

% Alias for the Solution section header
\newcommand{\solution}{\textbf{\large Solution}}

% Probability commands: Expectation, Variance, Covariance, Bias
\newcommand{\E}{\mathrm{E}}
\newcommand{\Var}{\mathrm{Var}}
\newcommand{\Cov}{\mathrm{Cov}}
\newcommand{\Bias}{\mathrm{Bias}}

\begin{document}
% \maketitle
% \pagebreak

\begin{homeworkProblem}
    \part

    Per dimostrare che $U(t, t') = \exp\left(-\frac{i}{\hbar} H_0(t- t')\right)$ soddisfa l'equazione di Schr{\"o}dinger data un'Hamiltoniana $H(t) = H_0$ indipendente dal tempo, iniziamo sostituendo $H(t)$ e $U(t, t')$ nell'equazione.

    \begin{align}\label{shrod_eq}
        i \hbar \pderiv{t}{U(t, t')} = H_0 U(t, t') \\
        i \hbar \pderiv{t}{\exp\left(- \frac{i}{\hbar} H_0(t - t')\right)} = H_0 U(t, t')
    \end{align}

    Calcoliamo la derivata parziale rispetto al tempo dell'esponenziale di matrice usando la sua definizione, cioè
    \begin{equation}\label{matrix_exp}
        \exp(A) = \sum_{n = 0}^{\infty} \frac{1}{n!} A^n = \mathds{1} + \sum_{n = 1}^{\infty} \frac{1}{n!} A^n
    \end{equation}

    Considerando solo la derivata parziale e applicando \eqref{matrix_exp} si ottiene

    \begin{align*}
        & \pderiv{t}{\left[\mathds{1} + \sum_{n = 1}^{\infty} \frac{1}{n!} \left(-\frac{i}{\hbar} H_0 (t - t')\right)^n \right]} =
        \\
        &= \sum_{n = 1}^{\infty} \frac{1}{n!} \left(-\frac{i}{\hbar} H_0\right)^n \pderiv{t}{(t - t')^n}
        \\
        &= \sum_{n = 1}^{\infty} \frac{1}{n!} \left(-\frac{i}{\hbar} H_0\right)^n n(t - t')^{n-1}
        \\
        &= -\frac{i}{\hbar} H_0 \sum_{n = 1}^{\infty} \frac{1}{(n-1)!} \left(-\frac{i}{\hbar} H_0\right)^{n-1} (t - t')^{n-1}
        \\
        &= -\frac{i}{\hbar} H_0 \sum_{n = 0}^{\infty} \frac{1}{n!} \left(-\frac{i}{\hbar} H_0\right)^n (t - t')^n
        \\
        &= -\frac{i}{\hbar} H_0 \exp\left(- \frac{i}{\hbar} H_0(t - t')\right) = -\frac{i}{\hbar} H_0 U(t, t')
    \end{align*}

    Sostituendo questo risultato in \eqref{shrod_eq} troviamo
    \begin{align*}
        i \hbar (-\frac{i}{\hbar} H_0) U(t, t') &= H_0 U(t, t') \\
        H_0 U(t, t') &= H_0 U(t, t')
    \end{align*}
    e quindi abbiamo dimostrato che l'eq. di Schr{\"o}dinger è soddisfatta.
    \\

    \part

    Possiamo sfruttare l'identità $\exp(A)\exp(B) = \exp(A + B)$, che vale però solo quando $A$ e $B$ commutano. Nel nostro caso questa condizione è soddisfatta.
    \begin{align*}
        & U(t, t')U(t', t'') =
        \\
        &= \exp\left(- \frac{i}{\hbar} H_0(t - t')\right) \exp\left(- \frac{i}{\hbar} H_0(t' - t'')\right)
        \\
        &= \exp\left(-\frac{i}{\hbar} H_0(t - t') -\frac{i}{\hbar} H_0(t' - t'')\right)
        \\
        &= \exp\left(-\frac{i}{\hbar} H_0 [(t - t') + (t' - t'')]\right)
        \\
        &= \exp\left(-\frac{i}{\hbar} H_0 (t - t'')\right) = U(t, t'')
    \end{align*}

    \part

    \begin{align*}
        & U(t, t) =
        \\
        &= \exp\left(- \frac{i}{\hbar} H_0(t - t)\right)
        \\
        &= \exp\left(- \frac{i}{\hbar} H_0 \cdot 0\right) = \mathds{1}
    \end{align*}

    \part

    Per quest'ultima identità possiamo sfruttare la seguente proprietà dell'esponenziale di matrice: \linebreak $(\exp{A})^\dagger = \exp\left(A^\dagger\right)$.
    Inoltre sfruttiamo il fatto che l'operatore Hamiltoniano $H_0$ è necessariamente hermitiano, quindi vale $H_0^{\dagger} = H_0$.
    \begin{align*}
        & U(t, t')U(t, t')^\dagger =
        \\
        &= \exp\left(- \frac{i}{\hbar} H_0(t - t')\right) \exp\left(- \frac{i}{\hbar} H_0(t - t')\right)^\dagger
        \\
        &= \exp\left(- \frac{i}{\hbar} H_0(t - t')\right) \exp\left(\left(- \frac{i}{\hbar} H_0(t - t')\right)^\dagger\right)
        \\
        &= \exp\left(- \frac{i}{\hbar} H_0(t - t')\right) \exp\left(\frac{i}{\hbar} H_0^\dagger(t - t')\right)
        \\
        &= \exp\left(- \frac{i}{\hbar} H_0(t - t')\right) \exp\left(\frac{i}{\hbar} H_0(t - t')\right)
        \\
        &= \exp\left(\mathbf{0}\right) = \mathds{1}
    \end{align*}
\end{homeworkProblem}

\begin{homeworkProblem}
    \[
        \exp\left(-\frac{i}{2} \phi \vec{n} \cdot \vec{\sigma}\right) = \mathds{1} \cos(\phi / 2) - i \vec{n} \cdot \vec{\sigma} \sin(\phi / 2)
    \]
    Per dimostrare questa identità è utile la seguente proprietà delle matrici di Pauli
    \begin{equation}\label{pauli_square}
        \left(\vec{n} \cdot \vec{\sigma}\right)^2 = \mathds{1}
    \end{equation}

    \`E evidente che se vale \eqref{pauli_square} valgono anche le seguenti due proprietà:
    \[
        \begin{aligned}
            \left(\vec{n} \cdot \vec{\sigma}\right)^{2n} &= \mathds{1} \\
            \left(\vec{n} \cdot \vec{\sigma}\right)^{2n + 1} &= \vec{n} \cdot \vec{\sigma}
        \end{aligned}
        \quad \forall n \in \mathbb{N}
    \]

    Dimostro \eqref{pauli_square} prima scrivendo in maniera esplicita il prodotto $\vec{n} \cdot \vec{\sigma}$. Chiamo $n_x$, $n_y$ e $n_z$ le componenti di $\vec{n}$, quindi
    $\vec{n} = (n_x, n_y, n_z)$.
    \[
        n_x
        \left(\begin{matrix}
            0 & 1 \\
            1 & 0
        \end{matrix}\right)
        + n_y
        \left(\begin{matrix}
            0 & -i \\
            i & 0
        \end{matrix}\right)
        +n_z
        \left(\begin{matrix}
            1 & 0 \\
            0 & -1
        \end{matrix}\right)
        = \left(\begin{matrix}
            n_z & nx - i n_y \\
            n_x + i n_y & -n_z
        \end{matrix}\right)
    \]
    Elevando al quadrato si ottiene
    \[
        \left(\begin{matrix}
            n_z & nx - i n_y \\
            n_x + i n_y & -n_z
        \end{matrix}\right) ^ 2
        = \left(\begin{matrix}
            n_x^2 + n_y^2 + n_z^2 & 0 \\
            0 & n_x^2 + n_y^2 + n_z^2
        \end{matrix}\right)
    \]
    Siccome $\vec{n}$ è unitario per ipotesi, significa che $n_x^2 + n_y^2 + n_z^2 = 1$. Quindi l'ultima matrice è $\mathds{1}$.
    Se ora applichiamo \eqref{matrix_exp} a $\exp\left(-\frac{i}{2} \phi \vec{n} \cdot \vec{\sigma}\right)$ troviamo

    \[
        \sum_{n = 0}^{\infty} \frac{1}{n!}\left(-\frac{i}{2} \phi \vec{n} \cdot \vec{\sigma} \right)^n
    \]
    Possiamo separare la sommatoria in due sommatorie dove la prima contiene solo potenze pari mentre la seconda solo quelle dispari
    \[
        \sum_{n = 0}^{\infty} \frac{1}{(2n)!}\left(-\frac{i}{2} \phi \vec{n} \cdot \vec{\sigma} \right)^{2n} +
        \sum_{n = 0}^{\infty} \frac{1}{(2n + 1)!}\left(-\frac{i}{2} \phi \vec{n} \cdot \vec{\sigma} \right)^{2n + 1}
    \]
    Con alcuni passaggi algebrici si ottiene
    \begin{align*}
        &\sum_{n = 0}^{\infty} \frac{1}{(2n)!}\left(-\frac{i}{2} \phi \vec{n} \cdot \vec{\sigma} \right)^{2n} +
        \sum_{n = 0}^{\infty} \frac{1}{(2n + 1)!}\left(-\frac{i}{2} \phi \vec{n} \cdot \vec{\sigma} \right)^{2n + 1} =
        \\
        &= \sum_{n = 0}^{\infty} \frac{1}{(2n)!}\left(-\frac{i}{2} \phi\right)^{2n} (\vec{n} \cdot \vec{\sigma})^{2n} +
        \sum_{n = 0}^{\infty} \frac{1}{(2n + 1)!}\left(-\frac{i}{2} \phi\right)^{2n + 1} (\vec{n} \cdot \vec{\sigma})^{2n + 1}
        \\
        &= \sum_{n = 0}^{\infty} \frac{1}{(2n)!}\left(\frac{\phi}{2} \right)^{2n} \mathds{1} +
        \sum_{n = 0}^{\infty} \frac{-i}{(2n + 1)!}\left(\frac{\phi}{2} \right)^{2n + 1} \vec{n} \cdot \vec{\sigma}
        \\
        &= \mathds{1} \sum_{n = 0}^{\infty} \frac{1}{(2n)!}\left(\frac{\phi}{2} \right)^{2n} -
        i \vec{n} \cdot \vec{\sigma} \sum_{n = 0}^{\infty} \frac{1}{(2n + 1)!}\left(\frac{\phi}{2} \right)^{2n + 1}
    \end{align*}

    Le due sommatorie rimaste sono le serie di Taylor di $\cos(\phi / 2)$ e $\sin(\phi / 2)$ rispettivamente, quindi l'identità è dimostrata.

\end{homeworkProblem}

\end{document}
